\documentclass[twoside]{article}

\usepackage{amsmath,amsthm,amssymb,epsfig}
\usepackage{hyperref}
\usepackage[numbers]{natbib}
\usepackage{textcomp}
\usepackage{listings}
\usepackage [english]{babel}
\usepackage [autostyle, english = american]{csquotes}
\MakeOuterQuote{"}

\theoremstyle{definition}
\newtheorem{thm}{Theorem}[section]
\newtheorem{lem}[thm]{Lemma}
\newtheorem{prop}[thm]{Proposition}
\newtheorem{cor}[thm]{Corollary}
\newenvironment{pf}{{\noindent\sc Proof. }}{\qed}
\newenvironment{map}{\[\begin{array}{cccc}} {\end{array}\]}

\DeclareMathOperator*{\argmin}{arg\,min}
\DeclareMathOperator*{\argmax}{arg\,max}
\newcommand{\unif}{\operatorname{unif}}

\newcommand{\comment}[1]{}
\theoremstyle{definition}
\newtheorem*{defn}{Definition}
\newtheorem*{exmp}{Example}
\newtheorem*{prob}{Problem}

\theoremstyle{remark}
\newtheorem*{rem}{Remark}
\newtheorem*{note}{Note}
\newtheorem*{exer}{Exercise}

\setlength{\oddsidemargin}{0.25 in}
\setlength{\evensidemargin}{-0.25 in}
\setlength{\topmargin}{-0.6 in}
\setlength{\textwidth}{6.5 in}
\setlength{\textheight}{8.5 in}
\setlength{\headsep}{0.75 in}
\setlength{\parindent}{0 in}
\setlength{\parskip}{0.1 in}
\everymath{\displaystyle}

\newcommand{\widgraph}[2]{\includegraphics[keepaspectratio,width=#1]{#2}}
\newcommand{\widgraphr}[3]{\includegraphics[keepaspectratio,width=#1,angle=#3]{#2}}

\newcommand{\lecture}[4]{
   \pagestyle{myheadings}
   \thispagestyle{plain}
   \newpage
   \setcounter{page}{1}
   \noindent
   \begin{center}
   \framebox{
      \vbox{\vspace{2mm}
    \hbox to 6.28in { {\bf ML Reading Group (Fall 2015) \hfill Lecture: #3} }
       \vspace{6mm}
       \hbox to 6.28in { {\Large \hfill #1  \hfill} }
       \vspace{6mm}
       \hbox to 6.28in { {\it Lecturer: #2 \hfill} }
      \vspace{2mm}}
   }
   \end{center}
   \markboth{#1}{#1}
   \vspace*{4mm}
}


%%%%%%%
% Some commonly used notation
%%%%%%%

\def\R{{\mathbb R}}
\def\X{{\mathcal X}}
\def\Y{{\mathcal Y}}
\def\E{{\mathbb E}}
\def\pr{{\mathbb P}}
\def\sign{{\rm sign}}

\newcommand{\percent}{$\%$}

\begin{document}

\lstset{
  basicstyle=\ttfamily,
  mathescape
}

\lecture{Interesting Linear Algebra Problem}{Jonathon Cai}{1}

\section{Today}

-- Compute linear algebra arithmetic using theory

\section{Theory}

\textbf{Theorem} 2.11 from Friedberg, which is true by the definitions of composition of functions and matrix multiplication:

Let $V, W$, and $Z$ be finite-dimensional vector spaces with ordered bases
$\alpha, \beta,$ and $\gamma$, respectively. Let $T: V \to W$ and $U: W \to Z$ be
linear transformations. Then

\[
  [UT]^\gamma_\alpha = [U]^\gamma_\beta[T]^\beta_\alpha
\]

An explicit example of the above is the following:

Consider $U(f(x)) = f'(x)$ where $U:P_3(\R) \to P_2(\R)$ and $T(f(x)) = \int_0^x f(t) dt$ where $T:P_2(\R) \to P_3(\R)$. $UT = I$, by calculus. 

Let the standard bases of $P_3(\R)$ and $P_2(\R)$ be $\alpha$ and $\beta$ respectively. 
As an easy exercise, verify \textbf{Theorem} 2.11 with the above example.

\textbf{Theorem} 2.23 from Friedberg:

Let $T$ be a linear operator on a finite-dimensional vector space $V$, and let
$\beta$ and $\beta'$ be ordered bases for $V$. Suppose that $Q$ is the change
of coordinate matrix that changes $\beta'$-coordinates into $\beta$-coordinates.
In other words, $Q = [I]^\beta_{\beta'}$.

Then

\[
[T]_{\beta'} = Q^{-1}[T]_\beta Q
\]

\textbf{Proof}:

Let $I$ be the identity transformation on $V$. Then $T = IT = TI$. So, by Theorem 2.11,

$Q[T]_{\beta'} = [I]^\beta_{\beta'}[T]^{\beta'}_{\beta'} = [IT]^{\beta}_{\beta'} = [TI]^\beta_{\beta'} = [T]^\beta_{\beta} [I]^\beta_{\beta'} = [T]_\beta Q$.

Now multiply the left hand side of the first and last terms by $Q^{-1}$. 

A subtle detail of this proof is that $Q$ is \emph{always} invertible. Why is this the case? Refer to \url{http://math.stackexchange.com/questions/25779/rigorously-proving-that-a-change-of-basis-matrix-is-always-invertible} for a great explanation.


\section{Problem}

Calculate $Q^{-1} A Q$ without inverting $Q$, entirely by hand, where

$Q = 
\begin{bmatrix}
    0 & 0 & 0 & 1 & 0 \\
    1 & 0 & 0 & 0  & 0 \\
    0 & 0 & 0 & 0 & 1\\
    0 & 1 & 0 & 0 & 0\\
    0 & 0 & 1 & 0 & 0
\end{bmatrix}
$ and $A = 
\begin{bmatrix}
    1 & 1 & 1 & 1 & 1 \\
    0 & 1 & 1 & 1 & 1 \\
    1 & 0 & 1 & 1 & 1 \\
    1 & 1 & 0 & 1 & 1 \\
    1 & 1 & 1 & 0 & 1 
\end{bmatrix}
$.

\subsection{Solution}

Recognize that $Q$ is the change of basis matrix where $\epsilon$ is the standard basis
of $\R^5$ and $\gamma = \{e_2, e_4, e_5, e_1, e_3\} = \{\gamma_1, \gamma_2, \gamma_3, \gamma_4 \gamma_5\}$. Consequently we may write $Q$ as
$[I]^{\epsilon}_{\gamma}$. By theorem 2.23, $Q^{-1} A Q = [A]_{\gamma}$.

So, we would like to compute $A \gamma_1, \ldots, A \gamma_5$ with respect to $\gamma$.
In order to compute this easily, recognize that $A \gamma_1 = A e_2$. Multiplying, note
we get the second column of $A$, or $c - e_3$ where $c = e_1 + e_2 + e_3 + e_4 + e_5$.
Clearly, $c = \gamma_1 + \ldots + \gamma_5$ as well. And remember that $e_3 = \gamma_5$.
So, $A\gamma_1 = c - \gamma_5$. Repeating for the others, we get the following matrix:

$[A]_{\gamma} = 
\begin{bmatrix}
    1 & 1 & 1 & 0 & 1 \\
    1 & 1 & 1 & 1 & 0 \\
    1 & 0 & 1 & 1 & 1 \\
    1 & 1 & 1 & 1 & 1 \\
    0 & 1 & 1 & 1 & 1 
\end{bmatrix}$.

\subsection{Alternate Solution}

Observe that $Q$ is orthogonal. So $Q^{-1} = Q^T$. Also observe that $Q$ and $Q^T$ are compositions of elementary row and composition operations.
Consequently, you can compute $Q^T A Q$ without performing a single matrix multiplication -- it can be done solely using row/column switches!

\end{document}
